\documentclass[a4paper,14pt,russian]{article}

\usepackage[english, russian]{babel}
\usepackage[T2A]{fontenc}
\usepackage{cmap} % для кодировки шрифтов в pdf
\usepackage[utf8]{inputenc}
\sloppy

\usepackage{graphicx}
\graphicspath{{img/}}
\usepackage{amstext, amssymb, amsfonts, amsmath, amsthm}
\usepackage{indentfirst} % отделять первую строку раздела абзацным отступом тоже
\usepackage[usenames,dvipsnames]{color} % названия цветов
\usepackage{amssymb}
\usepackage{listings}
\usepackage{caption}
\bibliographystyle{unsrt}

\linespread{1.3} % полуторный интервал
%\renewcommand{\rmdefault}{ftm} % Times New Roman
\frenchspacing
\renewcommand\contentsname{Projects List} %%% renaming the Table of Contents

%%%%%%%%%%%%
% страницы
% \captionsetup{figurewithin=section}
\renewcommand{\thefigure}{\arabic{figure}}
\usepackage{fancyhdr}
\pagestyle{fancy}
\fancyhf{}
\fancyfoot[R]{\thepage}
\fancyfoot[L]{CC BY-SA}
\fancyheadoffset{0mm}
\fancyfootoffset{0mm}
\setlength{\headheight}{17pt}
\renewcommand{\headrulewidth}{0pt}
\renewcommand{\footrulewidth}{0pt}
\fancypagestyle{plain}{
\fancyhf{}
\rhead{\thepage}}
\setcounter{page}{2} % начать нумерацию страниц с №

%%%%%%%%%%%%
% картинки
% см habrahabr.ru/post/144648/
\usepackage[tableposition=top]{caption}
\usepackage{subcaption}
\DeclareCaptionLabelFormat{gostfigure}{Рисунок #2}
\DeclareCaptionLabelFormat{gosttable}{Таблица #2}
\DeclareCaptionLabelSeparator{gost}{~---~}
\captionsetup{labelsep=gost}
\captionsetup[figure]{labelformat=gostfigure}
\captionsetup[table]{labelformat=gosttable}
\renewcommand{\thesubfigure}{\asbuk{subfigure}}
\usepackage{wrapfig}

%%%%%%%%%%%%%
% геометрия
\usepackage{geometry}
\geometry{left=2.5cm}
\geometry{right=1.5cm}
\geometry{top=2.0cm}
\geometry{bottom=2.0cm}

\setcounter{section}{0}

\begin{document}

\section {Введение}

Позитроний является одним из наиболее удобных объектов для теоретического и экспериментального изучения релятивистского связанного состояния. Благодаря малой массе его составляющих эффекты сильного и слабого взаимодействия пренебрежимо малы по сравнению с точностью современных экспериментов по спектроскопии позитрония. В то же время эти эксперименты имеют достаточную точность для сравнения с результатами современных теоретических исследований. Таким образом, можно и нужно найти с этой точностью уровни энергии и время жизни позитрония в рамках кэд.

Интерес к позитронию на протяжении долгого времени был связан как с проблемой разработки общих методов описания релятивистского связанного состояния, так и с наличием заметного расхождения теоретического предсказания для времени жизни ортопозитрония с экспериментальными данными. В настоящее время наиболее точный экспериментальный результат отличается от теоретического предсказания на 6 стандартных отклонений.

были вычислены \cite{a} два вклада, показывающих причины этого отклонения. Это, во-первых, вклад, возникающий при возведении в квадрат однопетлевой поправки к амплитуде аннигиляции. Основная идея вычисления этого вклада состоит в непосредственном нахождении поправок к амплитуде, вместо поправок к ширине, вычислявшихся в предыдущих работах. Вторым из "независимо вычисляемых" вкладов был найден вклад, связанный с поляризацией вакуума. Соответствующие результаты опубликованы в работах \cite{b} \cite{c}

Их результаты были подтверждены в \cite{d} \cite{e}

В настоящее время наиболее точно измеренным свойством позитрония является сверхтонкое расщепление его основного состояния, т.е. разница энергий 1351 и 115о-состояний. 

\newpage

\section{Тонкая структура}

Новым явлением оказывается тонкая структура спектра: разичие по ээнергии между уровнями с разными j, но с тем же самым значением n. Например, для Z = 1 $\Delta E = 10.9 [GHz]$

Можно показать, что тонкое расщепление является следствием спин-орбитальной связи:

$
\delta E = \Big( \cfrac {Z \alpha} {4m^2} \; \cfrac {\sigma L} {r^2} \Big)
$

В первом приближении пренебрегают магнитным полем, создаваемым спином ядра. Взаимодействие магнитным моментов ядра и электрона расщепляет уровень на дублеты.

Поправки: возбуждённые состояния атомов являются нестабильными, и атому могут спонтанно переходить в более низкое энергитическое состояние. В нерелятивистском дипольном приближении вероятность того, что за единицу времени произойдёт переход между двумя состояниями $\lambda$ и $\mu$, задаётся выражением

$
W_{\mu \leftarrow \lambda} = \cfrac {4} {3} \; \cfrac {(E_{\lambda} - E_{\mu})^3} {2j_{\lambda} + 1} \mid < \mu \| D \| \lambda >\mid^2
$

где  $\mid < \mu \| D \| \lambda >\mid $- приведённый матричный элемент диполоьного оператора $\vec{D} = e \vec{r}$

Во-вторых, заряженные частицы взаимодействуют с флуктуациями квантованного э/м поля, которое равно нулю лишь в среднем. Поэтому энергитические уровни оказываются смещёнными, что называется Лэмбовским сдвигом.


\subsection{Влияние ядра}

Ядро имеет конечные размеры, его заряд не сосредоточен в одной точке. Это сказывается преимущественно на $s$-состояниях, поскольку волновые функции с более высокими значениями $l$ в начале координат равны нулю.

\subsection{Итоговые вычисления}
В конце концов, результирующее уравнение, показывающее сдвиг энергии будет записано 

$
\Delta E = \cfrac {1} {2} \alpha^2 Ryd \Big[ \cfrac {7} {3} - (\cfrac {32} {9} + 2 \ln 2) \cfrac {\alpha} {\pi} \Big]
$

и будет называться формулой Карплуса - Клейна.

В этом уравнении учитываются
\begin{enumerate} 
\item вклад, возникающий при возведении в квадрат однопетлевой поправки к амплитуде аннигиляции.
\item вклад, связанный с поляризацией вакуума.
\item вклад радиационных поправок
\item вклад взаимодействия Брейта
\item вклад ложной инфракрасной расходимости
\item вклад аннигиляционного порядка
\item вклад двуфотонной аннигиляции.
\end{enumerate}
\newpage

\section{Практические измерения}

Состояния позитрония являются примером хорошего согласия, что как-то оправдывает теорию

Хотя позитроний представляет собой практически чисто электромагнитную систему, некоторые из развитых для него методов оказываются полезными и в других случаях, в таких, например, как модели связанных состояний кварков в адронах.

Разность энергий между более высоким триплетным (орто) и более низким синглетным (пара) основными состояниями позитрония, которые обозначают соответственно как $l^3S_1$ и $l^1S_0$, в настоящее время измерена с высокой точностью. Значения этого сверхтонкого расщепления $\Delta E_{ts}$ которые приводятся в литературе, равны

Миллз и Бирман:
$
\Delta E_{ts} = 2.033870 (16) \cdot 10^5 [MHz]
$

Иган, Фриз, Хьюдж и Ям:

$
\Delta E_{ts} = 2.033849 (12) \cdot 10^5 [MHz]
$

Эту разницу иногда также называют тонкой структурой позитрония. Недавно \cite{f} Миллз, Берко и Кантер измерили расстояние между  триплетными возбужденными уровнями:

$
E(2^3S_1) - E(2^3P_2) = 8624 \pm 2.8 [MHz]
$

Напомним, что все эти состояния нестабильны.

\newpage 
\begin{thebibliography}{0} % Список литературы
\bibitem{a}
А.П.Буриченко - Радиационные поправки второго порядка в позитронии, 2001, 62 стр. 
\bibitem{b}
А.П.Буриченко, ЯФ 56, вып.2, 123 (1993) Phys.At.Nucl. 56, 640 (1993)], Большой вклад в поправку ~ а2 к ширине ортопозитро-ния.
\bibitem{c}
А.П.Буриченко, Д.Ю.Иванов, ЯФ 58, 898 (1995), Phys.At.Nucl. 58, 832 (1995)], Вклад поляризации вакуума в поправку ~ а2 к ширине позитрония.
\bibitem{d}
G.S.Adkins, Phys. Rev. Lett. 76, 4903 (1996), Analytic evaluation of the orthopositronium-to-three-photon decay amplitudes to one-loop order.
\bibitem{e}
G.S.Adkins, Y.Shiferaw, Phys. Rev. A52, 2442 (1995), Two-loop corrections to the orthopositronium and parapositronium decay rates due to vacuum polarization.
\bibitem{f}
Ициксон К., Зюбер Ж.-Б. Квантовая теория поля: Пер. с англ.—М.: Мир, 1984. — 400 с. — Т. 2.
\end{thebibliography}

\end{document}
