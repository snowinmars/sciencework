\documentclass[a4paper,14pt,russian]{article}

\usepackage[english, russian]{babel}
\usepackage[T2A]{fontenc}
\usepackage{cmap} % для кодировки шрифтов в pdf
\usepackage[utf8]{inputenc}
\sloppy

\usepackage{graphicx}
\graphicspath{{img/}}
\usepackage{amstext, amssymb, amsfonts, amsmath, amsthm}
\usepackage{indentfirst} % отделять первую строку раздела абзацным отступом тоже
\usepackage[usenames,dvipsnames]{color} % названия цветов
\usepackage{amssymb}
\usepackage{listings}
\usepackage{caption}
\bibliographystyle{unsrt}

\linespread{1.3} % полуторный интервал
%\renewcommand{\rmdefault}{ftm} % Times New Roman
\frenchspacing
\renewcommand\contentsname{Projects List} %%% renaming the Table of Contents

%%%%%%%%%%%%
% страницы
% \captionsetup{figurewithin=section}
\renewcommand{\thefigure}{\arabic{figure}}
\usepackage{fancyhdr}
\pagestyle{fancy}
\fancyhf{}
\fancyfoot[R]{\thepage}
\fancyfoot[L]{CC BY-SA}
\fancyheadoffset{0mm}
\fancyfootoffset{0mm}
\setlength{\headheight}{17pt}
\renewcommand{\headrulewidth}{0pt}
\renewcommand{\footrulewidth}{0pt}
\fancypagestyle{plain}{
\fancyhf{}
\rhead{\thepage}}
\setcounter{page}{2} % начать нумерацию страниц с №

%%%%%%%%%%%%
% картинки
% см habrahabr.ru/post/144648/
\usepackage[tableposition=top]{caption}
\usepackage{subcaption}
\DeclareCaptionLabelFormat{gostfigure}{Рисунок #2}
\DeclareCaptionLabelFormat{gosttable}{Таблица #2}
\DeclareCaptionLabelSeparator{gost}{~---~}
\captionsetup{labelsep=gost}
\captionsetup[figure]{labelformat=gostfigure}
\captionsetup[table]{labelformat=gosttable}
\renewcommand{\thesubfigure}{\asbuk{subfigure}}
\usepackage{wrapfig}

%%%%%%%%%%%%%
% геометрия
\usepackage{geometry}
\geometry{left=2.5cm}
\geometry{right=1.5cm}
\geometry{top=2.0cm}
\geometry{bottom=2.0cm}

\setcounter{section}{0}

\begin{document}

\renewcommand\contentsname{Оглавление}
\tableofcontents
\vspace{\fill}

\newpage

\section{Введение} 

В релятивистском подходе связанные состояния и резонансы определяются по положению полюсов функций Грина. Простое обобщение уравнения Шредингера, к сожалению, невозможно, за исключением особых случаев, таких, как случай статических внешних источников, обсуждавшийся в связи с уравнением Дирака.

Вообще говоря, имеются трудности двух типов. Прежде всего необходимо учесть эффекты запаздывания, из-за которых в задачу вводится дополнительная переменная, а именно - относительное время. Альтернативное описание опирается на использование промежуточного поля. Однако при этом нельзя пренебрегать квантовыми свойствами последнего. Таким образом, оказывается, что само понятие связанного состояния двух тел является лишь результатом чрезмерного упрощения реальной ситуации. Несмотря на различные следствия, которые могут иметь большое практическое значение, во всех случаях, когда необходимо иметь последовательное описание, необходимо вернуться к общей теоретико-полевой картине. Это верно и тогда, когда мы хотим учесть высшие радиационные поправки.

Уравнение Бете-Солпитера описывает связанную систему состояний дву хтел (частиц) в КТП в  релятивистски ковариантной формализме. Уравнение было опубликовано в 1950 году в конце работы Yoichiro Nambu, но вывод был предоставлен позднее.

В КТП связанная система - это квантовомеханическое явление, при котором квантовые состояния двух объектов оказываются взаимозависимыми. Такая взаимозависимость сохраняется, даже если эти объекты разнесены в пространстве за пределы любых известных взаимодействий, что находится в логическом противоречии с принципом локальности. Суммируя все возможные взаимодействия, которые могут возникнуть между этими двумя телами бесконечное число раз, уравнение Бете-Солпитера является инструментом для расчета свойств связанных состояний и их решений.

\section{Вывод} 

Основным методом решения задач со взаимодействием, бесспорно, является теория возмущений, однако это далеко не единственный метод. Существуют, так называемые, непертурбативные методы и один из них ведет к уравнению Бете-Солпитера. Рассматривается система двух связанных [[фермион]]ов. В свободной теории, как известно, для одночастичной волновой функции $ \psi_{a} $  (где $a$ спинорный индекс) [[пропагатор]] определяется следующим образом:

$ \psi(x_{2})=-i\int{d\sigma{(x_{1})}S_{F}{(x_{2},x_{1})}n\!\!\!/(x_{1})\psi(x_{1})} $,

Тут используется запись с использованием «перечёркнутых матриц», $ n(x_{1}) $ — 4-х вектор внешней нормали. Интегрирование ведется по поверхности объема, включающего в себя событие $ x_{2} $, $ S_{F} $. — фейнмановский пропагатор.В случае невзаимодействующих частиц он определяется как решение следующего уравнения[1]:

$ (i\nabla\!\!\!/'-m_{0})S_{F}=\delta^{4}(x'-x)\qquad ( 1 ) $,

Аналогично пропагатору для одночастичной волновой функции, можно определить пропагатор для двучастичной волновой функции следующим выражением:

$ \psi_{ab}(x_{3},x_{4})=\int{d\sigma{(x_{1})}d\sigma{(x_{2})}S^{ab}{(x_{3},x_{1}x_{4},x_{2})}n\!\!\!/(x_{1})n\!\!\!/(x_{2})\psi_{ab}(x_{1},x_{2})} \qquad ( 2 ) $,

Здесь $ \psi_{ab} $  — спинор, обладающий двумя спинорными индексами $ a,b $. В случае невзаимодействующих частиц, двучастичная волновая функция распадается в произведение одночастичных,а пропагатор в произведение пропагаторов:

$ S^{0ab}(x_{3},x_{4};x_{1},x_{2})=iS_{F}^{a}(x_{3},x_{1})S_{F}^{b}(x_{4},x_{2}) $

Однако это самый тривиальный случай. Теперь же "включим" электромагнитное взаимодействие между двумя частицами. Если бы мы следовали идеологии теории возмущений, то получили бы, следуя Фейнману, $ S^{ab} $ представляется в виде:

$ S^{ab}(x_{3},x_{4};x_{1},x_{2})=iS_{F}^{a}(x_{3},x_{1})S_{F}^{b}(x_{4},x_{2})+\Sigma $

Под $ \Sigma $  понимается сумма всевозможных диаграмм, получаемых из теории возмущения. Основная идея, приводящая к уравнению заключается в том, что всю сумму диаграмм мы обозначаем, как некоторое ядро $ K $. Мы будем называть диаграмму приводимой, если  после удаления двух фермионных линий она становится несвязной. Тогда $ K $ можно представить в виде суммы двух вкладов: вклада приводимых диаграмм и вклада неприводимых диаграмм $ \overline{K} $. Можно показать[2], что выражение для $ S^{ab}(x_{3},x_{4};x_{1},x_{2}) $  может быть переписано как

$ S^{ab}(x_{3},x_{4};x_{1},x_{2})=iS_{F}^{a}(x_{3},x_{1})S_{F}^{b}(x_{4},x_{2})+\int{d^{4}x_{5}d^{4}x_{6}d^{4}x_{7}d^{4}x_{8}\cdot iS^{a}_{F}(x_{3},x_{5})iS^{b}_{F}(x_{4},x_{6})\overline{K}^{ab}(x_{5},x_{6};x_{7},x_{8})S^{ab}(x_{7},x_{8};x_{1},x_{2})} $

Подставляя это выражение в $ ( 2 ) $ получаем уравнение Бете-Солпитера:

$ \psi_{ab}(x_{1},x_{2})=\varphi_{ab}(x_{1},x_{2})+\int{d^{4}x_{3}d^{4}x_{4}d^{4} x_{5} d^{4}x_{6}\cdot iS^{a}_{F}(x_{1},x_{5})iS^{b}_{F}(x_{2},x_{6})\overline{K}^{ab}(x_{5},x_{6};x_{3},x_{4})\psi_{ab}(x_{3},x_{4})}\qquad ( 3 ) $

В этом выражении $ \varphi_{ab} $ — свободная двучастичная волновая функция, то есть волновая функция в отсутствии взаимодействия между частицами. Таким образом, получили интегральное уравнение Фредгольма II рода.

\section{Вывод для системы двух фермионов} 

Отправной точкой для вывода уравнения Бете-Солпитера является двух- или четырёхчастичное уравнение Дайсона:

$ G = S_1 S_2 + S_1 S_2  K_{12}  G  $

в импульсном пространстве, где "G" является двухчастичная функция Грина $ \langle\Omega| \phi_1  \phi_2  \phi_3   \phi_4 |\Omega\rangle $ $S$ - это свободный пропагатор и $K$ - это взаимодействие ядра, которое содержит все возможные взаимодействия между двумя частицами. Следующим шагом предположим, что связанные состояния проявляются в виде полюсов функции Грина. То есть две частицы собираются вместе и образуют связанное состояние с массой $М$, и это связанное состояние распространяется свободно, а затем расщепляется обратно в старые составные части. Поэтому вводится волновая функция Бете-Солпитера $ \Psi =  \langle\Omega| \phi_1  \phi_2|\psi\rangle $, которая является амплитудой перехода двух составляющих $ \phi_i $ в связанное состояние $ \Psi $. Затем предположим, что функция Грина в окрестности полюса будет иметь вид

$  G \approx \frac{\Psi\;\bar\Psi}{P^2-M^2},$

где $P$ - суммарный импульс системы. Видно, что если для этого импульса уравнение $ P ^ 2 = M ^ 2 $ имеет место, то это именно соотношение энергии-импульса Эйнштейна (с четыре-импульсом $ P_\mu = \left(E/c,\vec p  \right) $ и $ P^2 = P_\mu P^\mu $) для четырёхточечной функции Грина, содержащей полюс

$ \frac{\Psi\;\bar\Psi}{P^2-M^2} = S_1 S_2 +S_1 S_2  K_{12}\frac{\Psi\;\bar\Psi}{P^2-M^2} $

Сравнивая, получаем:

$ \Psi=S_1 S_2  K_{12}\Psi,   $

Это уже уравнение Бете-Солпитера, написанное в терминах волновых функций Бете-Солпитера. Можно ввести

$  \Psi = S_1 S_2 \Gamma $

и переписать уравнение в виде

$ \Gamma= K_{12} S_1 S_2 \Gamma $

В принципе ядро взаимодействия $К$ содержит все возможные двухчастичные неприводимые взаимодействий, которые могут возникнуть между двумя составляющими системы. В практических расчетах по очевидным причинам приходится выбрать лишь подмножество взаимодействий. Как и в квантовой теории поля, взаимодействие описывается с помощью обмена частицами (например, фотонами в квантовой электродинамике или глюонами в квантовой хромодинамике) самое простое взаимодействие описывается обменом только одной из этих частиц.

Так как уравнение Бете-Солпитера суммирует бесконечное число слагаемых, то диаграммы Фейнмана имеют вид лестницы.

\section{Ссылки}

[1]Walter Greiner, Joachim Reinhardt - Quantum Chromodynamics, 3, 2005, 46-47/475
[2]Walter Greiner, Joachim Reinhardt - Quantum Chromodynamics, 3, 2005, 347-348/475

\end{document}