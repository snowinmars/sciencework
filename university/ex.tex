\documentclass[a4paper,14pt,russian]{article}

\usepackage[english, russian]{babel}
\usepackage[T2A]{fontenc}
\usepackage{cmap} % для кодировки шрифтов в pdf
\usepackage[utf8]{inputenc}
\sloppy

\usepackage{graphicx}
\graphicspath{{img/}}
\usepackage{amstext, amssymb, amsfonts, amsmath, amsthm}

\usepackage{indentfirst} % отделять первую строку раздела абзацным отступом тоже
\usepackage[usenames,dvipsnames]{color} % названия цветов
\usepackage{amssymb}
\usepackage{listings}
\usepackage{caption}
\bibliographystyle{unsrt}

\linespread{1.3} % полуторный интервал
%\renewcommand{\rmdefault}{ftm} % Times New Roman
\frenchspacing
\renewcommand\contentsname{Projects List} %%% renaming the Table of Contents

%%%%%%%%%%%%
% страницы
% \captionsetup{figurewithin=section}
\renewcommand{\thefigure}{\arabic{figure}}
\usepackage{fancyhdr}
\pagestyle{fancy}
\fancyhf{}
\fancyfoot[R]{\thepage}
\fancyfoot[L]{CC BY-SA}
\fancyheadoffset{0mm}
\fancyfootoffset{0mm}
\setlength{\headheight}{17pt}
\renewcommand{\headrulewidth}{0pt}
\renewcommand{\footrulewidth}{0pt}
\fancypagestyle{plain}{
\fancyhf{}
\rhead{\thepage}}
\setcounter{page}{2} % начать нумерацию страниц с №

%%%%%%%%%%%%
% картинки
% см habrahabr.ru/post/144648/
\usepackage[tableposition=top]{caption}
\usepackage{subcaption}
\DeclareCaptionLabelFormat{gostfigure}{Рисунок #2}
\DeclareCaptionLabelFormat{gosttable}{Таблица #2}
\DeclareCaptionLabelSeparator{gost}{~---~}
\captionsetup{labelsep=gost}
\captionsetup[figure]{labelformat=gostfigure}
\captionsetup[table]{labelformat=gosttable}
\renewcommand{\thesubfigure}{\asbuk{subfigure}}
\usepackage{wrapfig}

%%%%%%%%%%%%%
% геометрия
\usepackage{geometry}
\geometry{left=2.5cm}
\geometry{right=1.5cm}
\geometry{top=2.0cm}
\geometry{bottom=2.0cm}


\setcounter{section}{0}

\begin{document}

\renewcommand\contentsname{Оглавление}
\tableofcontents
\vspace{\fill}

\newpage

\section{Предельный переход от уравнения Шредингера к уравнению Гамильтона-Якоби}

\subsection{Уравнение Шрёдингера}

Уравнение Шрёдингера — линейное дифференциальное уравнение в частных производных, описывающее изменение в пространстве (в общем случае, в конфигурационном пространстве) и во времени чистого состояния, задаваемого волновой функцией, в гамильтоновых квантовых системах. Играет в квантовой механике такую же важную роль, как второго закона Ньютона в классической механике или Эрвином Шрёдингером в 1925 году, опубликовано в 1926 году. Уравнение Шрёдингера не выводится, а постулируется методом аналогии с классической оптикой, на основе обобщения экспериментальных данных.

Уравнение Шрёдингера предназначено для частиц без спина, движущихся со скоростями много меньшими скорости света. В случае быстрых частиц и частиц со спином используются его обобщения (уравнение Клейна — Гордона, уравнение Паули, уравнение Дирака и др.)

В начале XX века учёные пришли к выводу, что между предсказаниями классической теории и экспериментальными данными об атомной структуре существует ряд расхождений. Открытие уравнения Шрёдингера последовало за революционным предположением де Бройля, что не только свету, но и вообще любым телам (в том числе и любым микрочастицам) присущи волновые свойства.

Исторически окончательной формулировке уравнения Шрёдингера предшествовал длительный период развития физики. Оно является одним из важнейших уравнений физики, объясняющих физические явления. Квантовая теория, однако, не требует полного отказа от законов Ньютона, а лишь определяет границы применимости классической физики. Следовательно, уравнение Шрёдингера должно согласовываться с законами Ньютона в предельном случае.

Средние значения механических величин для волнового пакета, который можно описать уравнением Шредингера, удовлетворяют классическим уравнениям Гамильтона (теорема Эренфеста).

Уравнение Шрёдингера инвариантно относительно преобразований Галилея. Из этого факта вытекает ряд важных следствий: существование ряда операторов квантовой механики, связанных с преобразованиями Галилея, невозможность описания состояний со спектром масс или нестабильные элементарные частицы в нерелятивистской квантовой механике (теорема Баргмана), существование квантовомеханических инвариантов, порождаемых преобразованием Галилея.

Уравнение Шрёдингера является более сложным по сравнению с уравнениями Гамильтона классической механики. Уравнения Гамильтона являются системой обыкновенных дифференциальных уравнений первого порядка, а уравнение Шрёдингера является дифференциальным уравнением в частных производных.

Уравнение Шрёдингера линейно, то есть если волновые функции 
$
\Psi
$
и 
$
\Phi
$
удовлетворяют уравнению Шрёдингера, то ему удовлетворяет любая их линейная комбинация 
$
\alpha \Psi + \beta \Phi
$
, где 
$
\alpha
$
и 
$
\beta
$ — комплексные числа. Вследствие этого линейная суперпозиция волновых функций не нарушается уравнением Шрёдингера и необходима операция измерения для редукции волновой функции. Линейность оператора Шрёдингера является следствием и обобщением принципа суперпозиции, который важен для корректной формулировки понятия операции измерения.

Уравнение Шрёдингера, как и уравнения Гамильтона, является уравнением первого порядка по времени. Оно является математическим выражением принципа статистического детерминизма в квантовой механике — данное состояние системы определяет её последующее состояние не однозначно, а лишь с определённой вероятностью, задаваемой при помощи волновой функции 
$
\Psi
$.

Уравнения Максвелла для электромагнитных волн в пустом пространстве можно путём введения новой комплексной величины 
$
\Psi = E + iH
$
, аналогичной волновой функции в уравнении Шрёдингера, преобразовать в одно уравнение 
$
i \frac{\partial  \Psi}{\partial t} = c \cdot \operatorname{rot} \Psi
$
, похожее на уравнение Шрёдингера.{{sfn|Мотт|с=21|1966}}

Уравнение Шрёдингера сходно с уравнениями теплопроводности и диффузии классической физики тем, что оно является уравнением первого порядка по времени и отличается от них наличием мнимого коэффициента перед 
$
\frac{\partial \Psi}{\partial t}
$. Благодаря ему оно может иметь и периодические решения.

Для всех квантовых систем, занимающих ограниченные области пространства, решения уравнения Шрёдингера существуют только для счётного множества значений энергии 
$
E_{n}
$
и представляют собой счётное множество волновых функций 
$
\Psi_{n}
$
, члены которого нумеруются набором квантовых чисел 
$
n
$.

Уравнение Шрёдингера можно получить из принципа наименьшего действия, рассматривая как уравнение Эйлера 

$
\frac{\partial L}{\partial \psi} - \sum_{k=0}^{3} \frac{\partial}{\partial x_{k}} \frac{\partial L}{\partial \left ( \frac{\partial \psi}{\partial x_{k}} \right )} = 0
$
некоторой вариационной задачи, в которой плотность лагранжиана имеет вид: 

$$
L=i \hbar \psi^{*} \frac{\partial \psi}{\partial t} - \frac{\hbar^{2}}{2m} \nabla \psi^{*} \nabla \psi - U(r,t) \psi^{*} \psi - i \hbar \frac{\partial \psi^{*}}{\partial t} \psi
$$.

Уравнение Шрёдингера не может объяснить спонтанного излучения, так как волновая функция возбуждённого состояния является точным решением зависящего от времени уравнения Шрёдингера.

Уравнение Шрёдингера не может описывать процесс измерения в квантовой механике, поскольку оно линейно, детерминистично и обратимо во времени, а процесс измерения нелинеен, стохастичен и необратим во времени.

Наиболее общая форма уравнения Шрёдингера — это форма, включающая зависимость от времени :

$$
i \hbar \frac{\partial}{\partial t}\Psi = \hat H \Psi, 
$$

где 
$
\hat H
$ — гамильтониан.

Пример нерелятивистского уравнения Шрёдингера в координатном представлении для точечной частицы массы 
$
m 
$
, движущейся в потенциальном поле c потенциалом 
$
 V(\vec{r} ,t) 
$ :

$$
i\hbar\frac{\partial}{\partial t} \Psi(\vec{r} ,t) = \left [ -\frac{\hbar^2}{2m}\nabla^2 + V(\vec{r},t)\right ] \Psi(\vec{r} ,t)
$$

В данном примере гамильтониан - это

$$
\hat{H} = -\frac{\hbar^2}{2m}\nabla^2 + V(\vec{r},t)
$$

\subsection{Переход}

Вид гамильтониана свободной частицы устанавливается ужеобщими требованиями, связаными с однородностью и изотропностью пространства ипринципом относительности Галлилея. В классической механике эти требования приводят к квадратичной зависимости энергии частицы от её импульса. В квантовой механие те же требования приводят к такому же соотношению доля собственных значений энергии и импульса.

Подставим в уравнение Шрёдингера  предельное выражение волновой функции

$
\Psi = ae^{\frac{iS} {\hbar}}
$
и,после дифференцирования, получим,что

$$
a \cfrac {\partial S} {\partial t} - i \hbar \cfrac {\partial a} {\partial t} + \cfrac {a} {2m} (\nabla S) ^2 - \cfrac {i \hbar} {2m} a \Delta S - \cfrac {i \hbar} {m} \nabla S \nabla a - \cfrac {\hbar} {2m} \Delta a + Ua = 0
$$

Приравняем по отдельности вещественные и мнимые члены к нулю.

$
\left \{ 
\begin{matrix}

\cfrac {\partial S} {\partial t} + \cfrac {1} {2m} (\nabla S) ^2 + U - \cfrac {\hbar^2} {2ma} \Delta a = 0

\\

\cfrac {\partial a} {\partial t} \cfrac {a} {2m} \Delta S + \cfrac {1}{m} \nabla S \nabla a = 0

\end{matrix}
\right. 
$

Из превого уравнения, пренебрегая членом с 
$
\hbar^2
$, получим классическое уравнение Гамильтона-Якоби для действия
$
S
$:

$$
\cfrac {\partial S} {\partial t} + \cfrac {1} {2m} (\nabla S)^2 + U = 0
$$

Второе уравнение после домножения на 
$
2a
$
может быть переписано в виде

$$
\cfrac {\partial a} {\partial t} + \mathrm{div} \Big( a^2 \cfrac {\nabla S} {m} \Big) = 0
$$

Это уравнение непрерывности, показывающее, что плотность вероятности
$
a
$
существует по законам классической механики.

\newpage

\section{Принцип Франка-Кондона}

\subsection{Определение}

Принцип Франка-Кондона - принцип в спектроскопии и квантовой химии, согласно которому безызлучательный перенос электрона может состояться только в том случае, когда его энергия в начальном и конечном состоянии равны. Существует несколько дополнительных формулировок этого принципа:

\begin{enumerate}
\item
Электроны не обмениваются энергией с ядрами.
\item
Электроны движутся гораздо быстрее, чем ядра.
\item
Электроны всегда имеют равновесную конфигурацию при любом расположении ядер.
\end{enumerate}

Электронные переходы происходят мгновенно по сравнению с временным масштабом ядерных движений, поэтому если молекула должна перейти на новый колебательный уровень через электронный переход, этот новый колебательный уровень должен быть мгновенно совместим с ядерными положений и импульсов колебательного уровня молекула в происходящей электронном состоянии. В полуклассической картине колебаний (колебаний) простого гармонического осциллятора, необходимые условия могут возникнуть в поворотных точках, где импульс равен нулю.

Классически, принцип Франка-Кондона является приближением: электронный переход происходит без изменений в позициях ядер в молекулярном субъекта и его окружения. Результирующее состояние называется состоянием Франка-Кондона. Квантово-механическая формулировка этого принципа заключается в том, что интенсивность вибронного перехода пропорциональна квадрату интеграла перекрытия между колебательными волновых функций двух состояний, которые участвуют в процессе перехода.

В приближении низких температур, молекула находится изначально в уровне v = 0 и при поглощении фотона необходимой энергии осуществляет переход в возбужденное электронное состояние. Электронная конфигурация нового состояния может привести к смещению положения равновесия ядер, составляющих молекулу. Вероятность того, что молекула может оказаться в каком-либо конкретном колебательном уровне пропорциональна квадрату перекрытия колебательных волновых функций исходного и конечного состояния. В электронно-возбужденных молекулах состояния быстро релаксируют до самого низкоэнергитичного колебательного уровня. Принцип Франка-Кондона применяется в равной степени как к поглощению, так и к флуоресценции.

\subsection{Квантово-механическая формулировка}

Рассмотрим электрический дипольный переход от начального колебательного состояния (
$
v
$) электронного уровня земли $(\varepsilon)$
$
|\varepsilon v\rangle
$
в какой-то колебательного состояния (
$
v
$) возбужденного электронное состояние $(\varepsilon)$
$
|\varepsilon v\rangle
$
Дипольный оператор молекулярного 
$
\mu
$
определяется зарядом (
$
-е
$
) и координатами (
$
r_i
$) электронов, а также зарядами (
$
+Z_je
$) и координатами (
$
R_j
$) ядер:

$$
\boldsymbol{\mu} = \boldsymbol{\mu}_e + \boldsymbol{\mu}_N = -e\sum\limits_i \boldsymbol{r}_i + e\sum\limits_j Z_j \boldsymbol{R}_j.
$$

Амплитуда вероятности
$
P
$
равна

$$
P = \left\langle \psi\right|\boldsymbol{\mu} \left| \psi \right\rangle =\int {\psi^*} \boldsymbol{\mu} \psi \,d\tau,
$$

где
$
\psi
$
и
$
\psi
$
, соответственно, общие волновые функции начального и конечного состояния. Общие волновые функции являются суперпозицией колебательных и электронных пространственных и спиновых волновых функций:

$
\psi = \psi_e \psi_v \psi_s. 
$

Такое разделение электронных и колебательных волновых функций является выражением приближения Борна-Оппенгеймера и является фундаментальным предположеним Франка-Кондона. Сочетание этих уравнений приводит к выражению для амплитуды вероятности с точки зрения отдельного электронного пространства, спина и колебательных вкладов:

$
P = \left\langle \psi_e \psi_v \psi_s \right| \boldsymbol{\mu} \left| \psi_e \psi_v \psi_s \right\rangle = \displaystyle\int \psi_e^* \psi_v^* \psi_s^* (\boldsymbol{\mu}_e + \boldsymbol{\mu}_N) \psi_e \psi_v \psi_s \,d\tau = \\
$

$
\hspace*{40pt} = \displaystyle \int \psi_e^* \psi_v^* \psi_s^* \boldsymbol{\mu}_e \psi_e \psi_v \psi_s \,d\tau + \int \psi_e^* \psi_v^* \psi_s^* \boldsymbol{\mu}_N \psi_e \psi_v \psi_s \,d\tau = \\
$

$
\hspace*{80pt} = \displaystyle \underbrace{\int \psi_v^* \psi_v \,d\tau_n}_{\text{Фактор} \atop \text{Франка-Кондона}}
 \underbrace{\int \psi_e^* \boldsymbol{\mu}_e \psi_e \,d\tau_e}_{\text{орбитальное} \atop \text {правило отбора}}
 \underbrace{\int \psi_s^* \psi_s \,d\tau_s}_{\text{спиновое} \atop \text{правило отбора}} +
 \underbrace{\int \psi_e^* \psi_e \,d\tau_e}_{= 0} \int \psi_v^* \boldsymbol{\mu}_N \psi_v \,d\tau_v \int \psi_s^* \psi_s \,d\tau_s.
$

Спин-независимая часть исходного интеграла здесь приближенно представлена как произведение двух интегралов

$$
\iint \psi_v^* \psi_e^* \boldsymbol{\mu}_e \psi_e \psi_v \,d\tau_e d\tau_n
\approx \int \psi_v^* \psi_v \,d\tau_n \int \psi_e^* \boldsymbol{\mu}_e \psi_e \,d\tau_e.
$$

\newpage

\section{Правила отбора}

Теория оболочечного строения ядра — одна из ядерно-физических моделей, объясняющих структуру атомного ядра. Она аналогична теории оболочечного строения атома. В оболочечной модели атома электроны наполняют электронные оболочки, и, как только оболочка заполнена, значительно понижается энергия связи для следующего электрона.

\subsection{Магические числа}
Аналогично в оболочечной модели ядра. При увеличении количества нуклонов (протонов или нейтронов) в ядре существуют определённые числа, при которых энергия связи следующего нуклона намного меньше, чем последнего. Особой устойчивостью отличаются атомные ядра, содержащие магические числа $2$, $8$, $20$, $50$, $82$, $114$, $126$, $164$ для протонов и $2$, $8$, $20$, $28$, $50$, $82$, $126$, $184$, $196$, $228$, $272$, $318$ для нейтронов.

Заметим, что оболочки существуют отдельно для протонов и нейтронов, так что можно говорить о «магическом ядре», в котором количество нуклонов одного типа является магическим числом, или о «дважды магическом ядре», в котором магические числа — количества нуклонов обоих типов. Из-за фундаментальных различий в заполнении орбит протонов и нейтронов дальнейшее заполнение происходит асимметрично: магическое числа для нейтронов $126$ и, теоретически, $184$, $196$, $228$, $272$, $318$, ... и только $114$, $126$ и $164$ для протонов. Этот факт имеет значение при поиске так называемых «островов стабильности». Кроме того, найдено несколько полумагических чисел, например, 
$
Z
$
=40 (
$
Z
$
— число протонов).

«Дважды магические» ядра — наиболее устойчивые изотопы, например, изотоп свинца $Pb-208$ с $Z=82$ и $N=126$ (N — число нейтронов).

Магические ядра являются наиболее устойчивыми. Это объясняется в рамках оболочечной модели: дело в том, что протонные и нейтронные оболочки в таких ядрах заполнены — как и электронные у атомов благородных газов.

\subsection{Теория}
Согласно этой модели, каждый нуклон находится в ядре в определённом индивидуальном квантовом состоянии, характеризуемом энергией, моментом вращения (его абсолютной величиной $j$, а также проекцией $m$ на одну из координатных осей) и орбитальным моментом вращения $l$.

Энергия уровня не зависит от проекции момента вращения на внешнюю ось. Поэтому в соответствии с принципом Паули на каждом энергетическом уровне с моментами $j$, $l$ может находиться ($2j + 1$) тождественных нуклонов, образующих «оболочку» ($j$, $l$). Полный момент вращения заполненной оболочки равен нулю. Поэтому, если ядро составлено только из заполненных протонных и нейтронных оболочек, то его спин будет также равен нулю.

Всякий раз, когда количество протонов или нейтронов достигает числа, отвечающего заполнению очередной оболочки (такие числа называются магическими), возникает возможность скачкообразного изменения некоторых характеризующих ядро величин (в частности, энергии связи). Это создаёт подобие периодичности в свойствах ядер в зависимости от $A$ и $Z$, аналогичной периодическому закону для атомов. В обоих случаях физической причиной периодичности является принцип Паули, запрещающий двум тождественным фермионам находиться в одном и том же состоянии. Однако оболочечная структура у ядер проявляется значительно слабее, чем в атомах. Происходит это главным образом потому, что в ядрах индивидуальные квантовые состояния частиц («орбиты») возмущаются взаимодействием («столкновениями») их друг с другом гораздо сильнее, чем в атомах. Более того, известно, что большое число ядерных состояний совсем не похоже на совокупность движущихся в ядре независимо друг от друга нуклонов, то есть не может быть объяснено в рамках оболочечной модели.

В этой связи в оболочечную модель вводится понятие квазичастиц — элементарных возбуждений среды, эффективно ведущих себя во многих отношениях подобно частицам. При этом атомное ядро рассматривается как ферми-жидкость конечных размеров. Ядро в основном состоянии рассматривается как вырожденный ферми-газ квазичастиц, которые эффективно не взаимодействуют друг с другом, поскольку всякий акт столкновения, изменяющий индивидуальные состояния квазичастиц, запрещён принципом Паули. В возбуждённом состоянии ядра, когда $1$ или $2$ квазичастицы находятся на более высоких индивидуальных энергетических уровнях, эти частицы, освободив орбиты, занимавшиеся ими ранее внутри ферми-сферы, могут взаимодействовать как друг с другом, так и с образовавшейся дыркой в нижней оболочке. В результате взаимодействия с внешней квазичастицей может происходить переход квазичастиц из заполненных состояний в незаполненное, вследствие чего старая дырка исчезает, а новая появляется; это эквивалентно переходу дырки из одного состояния в другое. Таким образом, согласно оболочечной модели, основывающейся на теории квантовой ферми-жидкости, спектр нижних возбуждённых состояний ядер определяется движением $1-2$ квазичастиц вне ферми-сферы и взаимодействием их друг с другом и с дырками внутри ферми-сферы. Этим самым объяснение структуры многонуклонного ядра при небольших энергиях возбуждения фактически сводится к квантовой проблеме 2—4 взаимодействующих тел (квазичастица — дырка или $2$ квазичастицы — $2$ дырки). Трудность теории состоит, однако, в том, что взаимодействие квазичастиц и дырок не мало, и потому нет уверенности в невозможности появления низкоэнергетического возбуждённого состояния, обусловленного большим числом квазичастиц вне ферми-сферы.

В других вариантах оболочечной модели вводится эффективное взаимодействие между квазичастицами в каждой оболочке, приводящее к перемешиванию первоначальных конфигураций индивидуальных состояний. Это взаимодействие учитывается по методике теории возмущений (справедливой для малых возмущений). Внутренняя непоследовательность такой схемы состоит в том, что эффективное взаимодействие, необходимое теории для описания опытных фактов, оказывается отнюдь не слабым. Кроме того, увеличивается число эмпирически подбираемых параметров модели. Также оболочечные модели модифицируются иногда введением различного рода дополнительных взаимодействий (например, взаимодействия квазичастиц с колебаниями поверхности ядра) для достижения лучшего согласия теории с экспериментом.

Оболочечная модель ядра фактически является полуэмпирической схемой, позволяющей понять некоторые закономерности в структуре ядер, но не способной последовательно количественно описать свойства ядра. В частности, ввиду перечисленных трудностей непросто выяснить теоретически порядок заполнения оболочек, а следовательно, и «магические числа», которые служили бы аналогами периодов таблицы Менделеева для атомов. Порядок заполнения оболочек зависит, во-первых, от характера силового поля, которое определяет индивидуальные состояния квазичастиц, и, во-вторых, от смешивания конфигураций. Последнее обычно принимается во внимание лишь для незаполненных оболочек. Наблюдаемые на опыте магические числа общие для нейтронов и протонов ($2$, $8$, $20$, $28$, $40$, $50$, $82$, $126$) отвечают квантовым состояниям квазичастиц, движущихся в прямоугольной или осцилляторной потенциальной яме со спин-орбитальным взаимодействием (именно благодаря ему и возникают числа $28$, $40$, $82$, $126$).

\newpage

\section{Правила отбора}

Правила отбора устанавливают допустимые квантовые переходы между уровнями энергии квантовой системы (атома, молекулы, кристалла, атомного ядра, элементарной частицы) при наложении на неё внешних возмущений. Если состояния системы характеризуются с помощью квантовых чисел, то правила отбора определяют их возможные изменения при квантовых переходах рассматриваемого типа. Математически правила отбора определяют отличные от нуля матричные элементы гамильтониана возмущённой системы в базисе собственных функций невозмущённой системы и являются следствием инвариантности гамильтониана (или лагранжиана) относительно преобразований группы симметрии системы и соответствующих сохранения законов. В частности, правила отбора для электрич. дипольных переходов в атоме или молекуле определяют ненулевые матричные элементы оператора взаимодействия дипольного момента системы m с электрич. вектором $Е$ электромагнитного поля в базисе собственных функций гамильтониана невозмущённой системы, а так как $Е$ не зависит от внутренних параметров системы, правила отбора определяют ненулевые матричные элементы дипольного момента системы. Правила отбора вводят и в случае приближённого описания системы; при этом они устанавливают, для каких переходов матричные элементы точного гамильтониана в базисе приближённых волновых функций отличны от нуля. 

Различают строгие и приближённые правила отбора. Квантовый переход называется запрещённым, если нарушается хотя бы одно правила отбора Строгие правила отбора обусловлены симметрией системы и строгими законами сохранения и налагают абсолютные запреты на квантовые переходы. Приближённые правила отбора характеризуют переходы между уровнями энергии, которые описываются приближёнными законами сохранения. Квантовое число полного углового момента атома ($J$) или молекулы ($F$)является точным, так как полный углового момент является инвариантом группы вращения, поэтому правила отбора для $J$ (или $F$) - строгие. В случае электрич. дипольных переходов возможны изменения квантовых чисел:
$
\Delta J = J - J ^\backprime = 0 \pm 1
$
в
$
\Delta M = M - M ^\backprime = 0 \pm 1
$
(где $J$, $J ^\backprime$ - квантовые числа полного момента атома в начальном и конечном состояниях; $М$, $М ^\backprime$ - квантовые числа проекций полных моментов).

В случае, когда не учитываются слабые взаимодействия, правила отбора по чётности состояний также являются строгими. Правила отбора нарушаются в сильных внешних полях за счёт поляризуемости атома или молекулы или при многофотонном поглощении.

Для атома существуют и другие строгие правила отбора. Для электрических переходов различной мультипольности изменение орбитального квантового числа 
$
\Delta l = 0, \pm 1, \ldots , \pm (\kappa) 
$
, для магнитных переходов
$
\Delta l = 0, \pm 1, \ldots , \pm (\kappa - 1)
$
. Для электрич. дипольных переходов
$
\Delta l = \pm 1
$
, т. е. такие переходы возможны между конфигурациями разл. чётности (правило Лапорта), а для электрических квадрупольных переходов
$
\Delta l = 0, \pm 2
$
. Правила отбора для проекции полного момента важны для определения поляризации спектральных линии испускания. 

В атомах, где осуществляется приближённый тип связи, квантовые переходы подчиняются приближённым правила отбора.

\end{document}